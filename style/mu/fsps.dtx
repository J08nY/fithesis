% \file{style/mu/fithesis-fsps.sty}
% \changes{v1.0.0}{2021/02/21}{Files were renamed after
%   breaking changes in package loading after PR 438 in \LaTeXe. [VN]}
% \changes{v1.0.0}{2021/03/19}{Updated urls to show the most recent
%   requirements and recommendations used in preparation of the
%   template. [TV]}
% This is the style file for the theses written at the Faculty of
% Sports Studies at the Masaryk University in Brno. It has been
% prepared in accordance with the formal requirements published at
% the website of the faculty\footnote{See \url{https://is.muni.cz/^^A
% auth/do/fsps/fak_predpisy/smernice-dekana/2020-03_Smernice_pokyn^^A
% y_vypracovani_ZP_bc-mgr-rig.pdf}}.
%    \begin{macrocode}
\NeedsTeXFormat{LaTeX2e}
\ProvidesPackage{fithesis/style/mu/fithesis-mu-fsps}[2021/04/24]
%    \end{macrocode}
% The file defines the color scheme of the respective faculty. Note
% the the color definitions are in RGB, which makes the resulting
% files generally unsuitable for printing.
%    \begin{macrocode}
\thesis@color@setup{
  links={HTML}{93BCF5},
  tableEmph={HTML}{A8BDE3},
  tableOdd={HTML}{EBEFF5},
  tableEven={HTML}{D1DAEB}}
%    \end{macrocode}
% The bibliography support is enabled. The |iso-authoryear| citations
% are used and the bibliography is sorted by name, title, and year.
%    \begin{macrocode}
\thesis@bibliography@setup{
  style=iso-authoryear,
  sorting=nyt}
\thesis@bibliography@load
%    \end{macrocode}
% The file loads the following packages:
% \begin{itemize}
%   \item\textsf{tikz} -- Used for dimension arithmetic.
%   \item\textsf{geometry} -- Allows for modifications of the type
%     area dimensions.
%   \item\textsf{setspace} -- Allows for line height modifications.
% \end{itemize}
% In addition to this, the type area width is set to
% 14\,cm in accordance with the formal requirements of the faculty.
%    \begin{macrocode}
\thesis@require{tikz}
\thesis@require{geometry}
\thesis@require{setspace}
\geometry{top=30mm,bottom=30mm,left=40mm,right=30mm,includeheadfoot}
%    \end{macrocode}
% The paragraph indentation is 1.25\,cm as per the requirements of the faculty.
%    \begin{macrocode}
\setlength{\parindent}{1.25cm}
%    \end{macrocode}
% \changes{v0.3.49}{2018/02/11}{Removed an extraneous \cs{vskip} in
%   the style files for the Masaryk University in Brno. [VN]}
% \changes{v1.0.0}{2021/02/26}{^^A
%   The style files for the Faculty of Sports Studies
%   at the Masaryk University in Brno no longer
%   redefine the \cs{thesis@blocks@titlePage@footer} macro, which
%   is no longer defined. [VN]}
% \begin{macro}{\thesis@blocks@frontMatter}
% The |\thesis@blocks@frontMatter| macro sets up the style of the
% front matter of the thesis. The leading is adjusted in
% accordance with the requirements of the faculty.
% \begin{macrocode}
\def\thesis@blocks@frontMatter{%
  \thesis@blocks@clear
  \pagestyle{plain}
  \parindent 1.5em
  \setcounter{page}{1}
  \pagenumbering{roman}
  \onehalfspacing}
%    \end{macrocode}
% \end{macro}\begin{macro}{\thesis@blocks@mainMatter}
% The |\thesis@blocks@mainMatter| macro sets up the style
% of the main matter of the thesis. The leading is adjusted in
% accordance with the requirements of the faculty.
% \begin{macrocode}
\def\thesis@blocks@mainMatter{%
  \thesis@blocks@clear
  \setcounter{page}{1}
  \pagenumbering{arabic}
  \pagestyle{thesisheadings}
  \parindent 1.5em
  \onehalfspacing}
%    \end{macrocode}
% \end{macro}\begin{macro}{\thesis@blocks@bibliography}
% The |\thesis@blocks@bibliography| macro typesets the
% bibliography. The leading is adjusted in accordance
% with the requirements of the faculty.
% \begin{macrocode}
\def\thesis@blocks@bibliography{%
  \ifthesis@bibliography@loaded@
    \ifthesis@bibliography@included@\else
      \singlespacing
      \thesis@blocks@clear
      {\emergencystretch=3em%
      \printbibliography[heading=bibintoc]}%
    \fi
  \fi}
%    \end{macrocode}
% \end{macro}\begin{macro}{\thesis@blocks@declaration}
% The |\thesis@blocks@declaration| macro typesets the declaration
% text. Unlike the generic |\thesis@blocks@declaration| macro from
% the \texttt{style/mu/fithesis-sci.sty} file, this definition
% includes the date and a blank line for the author's signature, as
% per the requirements of the faculty.
%
% Along with the macros required by the locale file interface, the
% locale files need to define the following macros:
% \begin{itemize}
%   \item|\thesis@|\textit{locale}|@authorSignature| -- The
%     label of the author's signature field
%   \item|\thesis@|\textit{locale}|@formattedDate| -- A
%     formatted date
% \end{itemize}
%    \begin{macrocode}
\def\thesis@blocks@declaration{%
  \thesis@blocks@clear
  \begin{alwayssingle}%
    \chapter*{\thesis@@{declarationTitle}}%
    \thesis@declaration
    \vskip 2cm%
    {\let\@A\relax\newlength{\@A}
      \settowidth{\@A}{\thesis@@{authorSignature}}
      \setlength{\@A}{\@A+1cm}
    \noindent\thesis@place, \thesis@@{formattedDate}\hfill
    \begin{minipage}[t]{\@A}%
      \centering\rule{\@A}{1pt}\\
      \thesis@@{authorSignature}\par
    \end{minipage}}
  \end{alwayssingle}}
%    \end{macrocode}
% \end{macro}
% Note that there is no direct support for the seminar paper and
% thesis proposal types.  If you would like to change the contents
% of the preamble and the postamble, you should modify the
% |\thesis@blocks@preamble| and |\thesis@blocks@postamble| macros.
%
% All blocks within the autolayout preamble and postamble that are
% not defined within this file are defined in the
% \texttt{style/mu/fithesis-base.sty} file.
% \changes{v1.0.0}{2021/03/22}{Reorganised the blocks to fit the
%   faculty's requirements. [TV]}
% \changes{v1.0.0}{2021/04/24}{^^A
%   Change \cs{thesis@facultyLogo},
%   \cs{thesis@blocks@facultyLogo@monochrome}, and
%   \cs{thesis@blocks@facultyLogo@color} to use the new logotype
%   of the Masaryk University in Brno in the correct size and
%   localization. Add \cs{thesis@blocks@seal}. [VN]}
%    \begin{macrocode}
\def\thesis@blocks@preamble{%
  \thesis@blocks@coverMatter
    \thesis@blocks@cover
  \thesis@blocks@frontMatter
    \thesis@blocks@titlePage
    \thesis@blocks@seal
    \thesis@blocks@bibEntry
    \thesis@blocks@abstract
    \thesis@blocks@bibEntryEn
    \thesis@blocks@abstractEn
    \thesis@blocks@thanks
    \thesis@blocks@tables}
\def\thesis@blocks@postamble{%
  \thesis@blocks@bibliography}
%    \end{macrocode}
